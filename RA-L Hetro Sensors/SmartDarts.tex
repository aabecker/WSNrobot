\section{Smart Darts}\label{sec:SmartDarts}

\subsection{Design}

The SmartDart combines a geophone (GS-100) with the fins and body of a lawn Jart\textsuperscript{TM}, using a 3D-printed chamber that encloses a WiFi-enabled micro-controller (Electron 2G, particle.io) as shown in Fig.~\ref{fig:Smart_Dart_overview}. 
The center of the chamber is slotted to fit a wooden plate holding an accelerometer that transmits data back to the user through the Photon. 
The centered accelerometer card allows placing the microcontroller and battery on opposite sides, centering the center of mass.
Designs and instructions to build a SmartDart at \cite{Victor2016Thingiverse}.



\begin{figure} \centering
{\includegraphics[width=\columnwidth]{Smart_Dart_overview.pdf}}
\caption{Cross-section of the SmartDart sensor. It consists of a lawn  Jart\textsuperscript{TM} fin, electron micro-controller, 3D printed protective casing and a geophone} 
\label{fig:Smart_Dart_overview}
\end{figure}

%%%%%%%%%%%%%%%%%%%%%%%%%%%%%%%%%%%%%%% 
\subsection{Experiments} 
The following sections compare SmartDart performance.
\subsubsection{ Drop tests in different soils}  
This experiment varied the drop height and measured the penetration depth in four types of soil. 
Proper planting of a geophone requires good contact with the soil, in a vertical position. 
To determine the minimum height of deployment for this desired planting, each trial measured the penetration depth and angular deviation from the vertical. 
This experiment compared drop tests as a function of soil type. 

To determine how smart darts perform in different soils, this experiment measured penetration into four soil types. Each trial was performed by holding the darts at the tip opposite to the spike in a vertical position and releasing them at varying heights into the buckets of soil and measuring their penetration depth, and angular deviation. To measure penetration depth, the buried darts were marked where the spike met the soil, the dart was then pulled from the soil, and the distance from the spike tip to the marking was measured with calipers. The angular deviation was recorded from the accelerometer inside the dart. The soil types were categorized by their compression strength, in kg/cm$^3$, measured using a soil pocket penetrometer (CertifiedMTP). Measurements for compression strength vary with small deviation in measurement location, so we repeated this measurement 10 times at 10 different locations in each soil type and took the average. These values for soil compression strength and a graph displaying heights vs. penetration depth are displayed in Fig.~\ref{fig:DepthPlotIndoors}, and a graph of angular deviation is in Fig.~\ref{fig:AnglePlotIndoors}. 


\begin{figure} \centering
{\includegraphics[width=\columnwidth]{indoor_depth_plot.pdf}}
\caption{Drop height vs. penetration depth in four soil types.} 
\label{fig:DepthPlotIndoors}
\end{figure}

\begin{figure} \centering
{\includegraphics[width=\columnwidth]{indoor_angle_plot.pdf}}
\caption{Drop height vs. angle of deviation in four soil types.} 
\label{fig:AnglePlotIndoors}
\vspace{-1em}
\end{figure}

\subsubsection{Straight vs. Bent Fins}

To determine the difference in performance between SmartDarts with straight fins and twisted fins, we ran a drop test with 10 trials for both types of dart at a constant height in one soil type. Each trial was initialized by holding the dart horizontally at a height of 10.5 meters, dropping it into the soil, and recording the penetration depth and angular deviation. Holding the darts horizontally emphasized the angle-correcting behavior of the fins. The penetration depth and angular penetration were measured and recorded as in the other drop test experiment in different soils. A graph showing the values recorded for penetration depth and angular deviation in Fig.~\ref{fig:StraightBentPic}  reveals that SmartDarts with twisted fins had less angular deviation, but also less penetration depth. 

\begin{figure} \centering
  {\includegraphics[width=\columnwidth]{StraightvsBent_pic.pdf}}
 \caption{Outdoor Drop test comparing Straight vs Bent fins performance.
 a.)  smart dart dropping 
 b.)  measuring drop height} 
 \label{fig:StraightBentPic}
 \vspace{-1em}
\end{figure}
\begin{figure} \centering
  {\includegraphics[width=\columnwidth]{StraightvsBent_depthangle.pdf}}
 \caption{\label{fig:StraightBentDepth}Straight vs Bent fins comparing a.) penetration depth b.) angle of deviation. Experiment used a fixed drop height of 9.8 m.} 
\end{figure}
%\begin{figure}\centering 
%\subfigure[\label{subfig:StraightBentDepth}]
%  {\includegraphics[width=.45\columnwidth]{StraightvsBent_depth.pdf}}
% \subfigure[\label{subfig:StraightBentAngle}]
%  {\includegraphics[width=.45\columnwidth]%{StraightvsBent_angle.pdf}}
%   \vspace*{-.1in}
 %\caption{Straight vs Bent fins comparing a.) Penetration Depth b.) Angle of Deviation. Experiment used a fixed drop height of 9.8 m. \label{fig:StraightBent}}
 %\vspace*{-.1in}
%\end{figure}
\subsubsection{Shot gather comparison}
Seismic explorations use thousands of geophones to conduct a seismic survey. 
This experiment compared the performance of a traditional cabled $24$ geophone system connected to a $24$ channel seismic recorder and a battery with readings from SmartDarts.
The geophones were planted vertically into the ground, three meters apart from one another.  
We used a vibrating truck setup to generate the seismic wave. 
Only four functional SmartDarts were built, so these four were dropped from the UAV, a seismic wave was generated and recorded, and the darts were redeployed.

Results of the seismic survey field test comparison between a $24$ channel traditional cabled geophone system and the SmartDarts are shown in Fig.~\ref{fig:shotgather_auto_drop}.  
Both plots were obtained using a \emph{Strata-Visor}, a device that can obtain, store and plot the sensed data. 
It is extensively used with traditional geophone setups because the geophones can only sense vibrational waves and are dependent on other devices for storage and data processing. 
To allow a fair comparison, the SmartDart's  ability to store sensed data was not used in this experiment. 
With the exception of SmartDart reading \#19, which may be due to poor terminal connections, the SmartDart data corresponds well to data from a traditional setup.


\begin{figure} \centering
  {\includegraphics[width=\columnwidth]{shotgather_auto_drop.pdf}}
 \caption{Shot gather comparison of traditional geophones vs autonomously dropped SmartDart sensors. a.) Traditional b.) SmartDarts} 
 \label{fig:shotgather_auto_drop}
\end{figure}


 