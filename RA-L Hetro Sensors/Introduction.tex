\section{Introduction}\label{sec:Introduction}
Seismic surveying is a geophysical technique involving sensor data collection and signal processing. It aims at identifying and retrieving hydrocarbons like coal, petrol, natural gas. Traditional seismic surveying involves manual laborers placing geophone sensors at specific locations connected by cables. Cables are bulky and the amount required is directly proportional to the area surveyed. On an average hundreds of square kilometers  would be surveyed and miles of cabling is required. Seismic surveying is done at remote locations and problems like accessibility, harsh  conditions and especially transportation of bulky cables and sensors phenomenally increases the cost. Nodal sensors are autonomous units that do not require bulky cabling. They have an internal seismic recorder, which is basically a micro-controller that controls and records seismic readings. This technology gets rid of bulky cabling and there by reduce the overall cost. Currently nodal sensors are becoming popular at USA due to reduced costs in seismic sensing.

