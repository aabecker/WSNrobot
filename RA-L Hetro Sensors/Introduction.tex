\section{Introduction}\label{sec:Introduction}
Seismic surveying is a geophysical technique involving sensor data collection and signal processing. 
It aims at identifying hydrocarbons reservoirs of coal, petrol, and natural gas. 
Traditional seismic surveying involves manual laborers repeatedly placing geophone sensors at specific locations connected by cables. 
Cables are bulky and the length required is proportional to the area surveyed. 
Surveys routinely cover hundreds of square kilometers, requiring kilometers of cabling. 
Remote locations often require seismic surveying, with concomitant problems of inaccessibility, harsh  conditions, and  transportation of bulky cables and sensors.  
These factors increases the cost. 

  Nodal sensors are a relatively new development to the seismic sensing.
  Nodal sensors are autonomous units that do not require bulky cabling. 
  They have an internal seismic recorder, a micro-controller that records seismic readings from a high-precision accelerometer. 
  Because this technology does not require cabling, the overall cost is reduced. 
  Nodal sensors are becoming popular due to reduced costs in seismic sensing.
  However, these sensors are still planted and recovered by hand.  

\begin{figure}
\centering
\begin{overpic}[width=\columnwidth]{intro.pdf}\end{overpic}
\caption{\label{fig:Hetero_overall}
The heterogeneous sensor system presented in this paper: wireless SmartDarts and a SeismicSpider, both designed for deployment from a UAV. 
}
\end{figure}


We propose a heterogeneous robotic system for obtaining seismic data, shown in Fig.~\ref{fig:Hetero_overall}. The system consists of two sensors, the SmartDart and  the SeismicSpider.  
The SmartDart is a dart-shaped wireless sensor that is planted in the ground by dropping from a UAV. 
The SeismicSpider is a mobile hexapod with three of legs replaced by geophones.
This system is designed to automate sensor deployment, minimizing cost and time while maximizing accuracy, repeatability, and efficiency.
  The technology presented may have wide applicability where quickly deploying sensor assets is essential, including geo-science, earthquake monitoring~\cite{dominici2012micro}, defense, and wildlife monitoring. \todo{add citations for each}
