 \section{Conclusion}\label{sec:Conclusion}
In this paper we present an autonomous technique for geophone placement, recording, and retrieval. This robot can enable automating a job that currently requires large teams of manual laborers.
 The paper described hardware experiments demonstrating the efficacy of this technique and comparing it with traditional manual techniques. The drone-sensing platform's output was comparable to a well planted geophone, suggesting the feasibility of the proposed system. Autonomous landing was displayed using GPS, thereby closing the control loop. This proved human involvement could be drastically minimized by adopting the proposed technique. Angle of penetration was compared between different soil types resulted in deviations of around $2$ deg. This proved the excellency in sensor platform design and reduced errors in sensor data. The system displayed the ability to penetrate soil types like sand and grass and, inability to penetrate hard types like dry clay yet it could perform sensing and obtain sensory data.
 
 In the future drone systems could be designed solely for seismic exploration purposes there by increasing robustness, increasing flight and stationary periods, and be weatherized.  
A quad rotor system in general has limitations in flight time and in the future we would like to separate the sensing platform from the deployment unit. By creating mechanisms for deployment i.e. drop and pick up of sensing units. Designs could be made to make sensors either immobile passive sensing units that just listen or mobile active units that could create and measure a seismic wave. Given a heterogeneous set of sensing unis, further optimization could give insight on number of each type of sensing unit required.  Pre and post signal processing techniques could be adapted to improve the quality of sensing. Data could be transmitted in real time to ease the exploration process and, identify errors and perform corrections. Transmitting high amounts of data is an issue, novel methods to compress, transmit, receive and interpret is an exciting research direction. Creating path planning algorithms for performing deployment retrieval tasks constrained by the availability of resource is an interesting future direction. It may be more beneficial to deploy one or more sensor packages and return the drone to a home base for charging or have a team of drones that divide the task of sensor deployment and retrieval. The precision on the GPS could be improved by upgrading to a RTK (real time kinematic) or a DGPS (Differential GPS) system. Including other sensors like lasers, sonar, or a vision system could improve precision in deployment, can improve robustness to random disturbances and avoid obstacles. There are many opportunities for future work. A mobile app could be created to perform tailor-made tasks focusing on seismic exploration, which could be used by an human/robot operator to plan an exploration strategy. These ideas could be enforced to make the real system accessible and operational by a minimal work force.

