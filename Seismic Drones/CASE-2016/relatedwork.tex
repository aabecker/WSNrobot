\section{Overview and Related Work}\label{sec:RelatedWork}
\begin{figure}
\centering
\begin{overpic}[width =\columnwidth]{overview.pdf}\end{overpic}
\caption{\label{fig:sensor_types}
 Comparing the state of art seismic survey sensors a.) A traditional cabled system, the three geophones(sensors) are connected in series to the seismic recorder and battery. b.) Autonomous nodal systems, each geohone has a seismic recorder and a battery there by making the system autonomous.} 
\end{figure}

\subsection{Seismic Exploration Methods}

%\todo{cite the book that Li sent us.  Also cite some papers by Rob.}
During seismic surveys a source is triggered to generate seismic/vibrational waves that propagate under the earth's surface and are sensed by geophone sensors and recorded for latr analysis to detect the presence of hydrocarbons. Fig.  \ref{fig:sensor_types} describes the current sensors available and the proposed solution, the seismic drone. These sensors are used to sense the vibrational wave that propagates with a velocity $c$ in the positive and negative $x$-directions and is represented by the equation 1, in 1-dimension.
\begin{equation}
\frac{\partial^{2}{U}}{\partial^{2}{t}} = {c}^{2}\frac{\partial^{2}{U}}{\partial^{2}{x}}
\end{equation}
The velocity of a seismic wave approximately ranges from $2-8$km/s.
Its general solution is given by
\begin{equation}
u(x,t) = f(x \pm ct)
\end{equation}
The equations stated above are generalized equations that represent a vibrational wave in physics, and can be used to describe a string that is vibrating. Hence we obtain the equation where $F$ is vibration force and $\rho$ is density.
\begin{equation}
{c}^{2} = F/\rho
\end{equation}
This equation is a hyperbolic equation from the theory of linear partial differential equations and is challenging to solve because of sharp features that can reflect off boundaries. 
The original seismic wave equation is a $3$-dimensional equation that scales in complexity and the link between force-displacement must be analyzed from a stress-strain relationship of an elastic solid. These equations can be found in many geophysics textbooks for example see ~\cite{shearer2009introduction}.

 \paragraph{Cabled Systems}

 Traditional \emph{cabled systems} are extensively used for seismic data acquisition in hydrocarbon explorations. A group of sensors (geophones) are connected to each other in series using long cables, and this setup is connected to a seismic recorder and a battery. The seismic recorder consists of a micro-controller which synchronizes the data acquired with a GPS signal and store it in the onboard memory. Generally four-cell Lithium Polymer (LiPo, 14.8V, 10Ahrs) batteries are used to power this system. This method of data acquisition requires many manual laborers and a substantial expenditure for transporting the cables. The major difficulties faced in using cabled system for data acquisition are (1.) Conducting a seismic survey in rugged terrains (2.) The manual labor available might be unskilled or expensive depending on the location.  
 
 \paragraph{Autonomous Nodal systems}

 Currently \emph{autonomous nodal systems}~\cite{wood1998distributed} are extensively used for conducting seismic data acquisition surveys in USA. Unlike traditional cabled systems, autonomous nodal systems are not connected using cables. The sensor, seismic recorder and battery are all combined into a single package called a node, and this node can autonomously record data. Even in these systems the data is stored in the on-board memory and can only be acquired after the survey is completed. This is disadvantageous since errors cannot be detected and rectified while conducting the survey. Recently, wireless autonomous nodes have been developed. These systems can transmit data wirelessly as a radio frequency in real time by ~\cite{jiang2015geophysical}. Yet these systems still require manual laborers for planting the autonomous nodes at specific locations and deploying large antennas which are necessary for wireless communication.
 
\subsection{Seismic Drone}  

The concept using robots to place seismic sensors is not new.  Postel et al. describe using mobile robots for geophone placement in patent application~\cite{DSSMaA14}.  Mobile robots have placed seismic sensors on the moon~\cite{LSisMSE81} and plans are underway for a swarm of seismic sensors for Mars exploration ~\cite{MAPL2006}.

This paper presents a flying seismic drone. Combines the quality of data acquisition present in a traditional exploration method with an autonomous unmanned air vehicle (UAV) which has high maneuverability and the capability of performing robust movements. A seismic recorder, battery, and four geophones are embedded onto a platform which can be attached to an UAV, the setup is shown in Fig.  \ref{fig:sensor_types}. By inputting a specific GPS location, the UAV can accurately deploy the seismic data acquisition system. The geophones obtain data which is processed by the seismic recorder and stored in the on-board memory. The major advantage of the drone is automating the deployment process and thereby eliminating humans from the loop. By using a robot to perform the above task, costs and errors are reduced. Since we use the same micro-controller as in the traditional cabled systems, we obtain the same 24-bit accuracy on the ADC conversion and sampled rates as low as half a millisecond. One current drawback of the seismic drone is that it cannot transmit data wirelessly and hence we cannot obtain seismic plots in real-time. Since the deployment is autonomous, it is precise and the system has the ability to re-deploy or return home from the current deployment site. 

\begin{figure}
\centering
\begin{overpic}[width =\columnwidth]{drone_base.pdf}\end{overpic}
\caption{\label{Sensor_Base}
The seismic drone's sensor base  consists of four geophones, a Seismic Recorder (SR) and a LiPo battery(14.8V, 0.5Ah, 4 cells).
}
\end{figure}
 
%\subsection{Seismic Wireless Sensor Network Drone}
 %  \todo{cite some recent robotics papers on drone sensor networks}
   
  % This system also employs an UAV for sensor deployment and it has the ability to transmit data wirelessly. In the current system, the data is transmitted via Bluetooth transmission and is limited to a range of 50 m. We use an Arduino Mega processor which possesses a 12-bit ADC and a maximum sampling rate of 1 ms. The proposed system was developed using commercially accessible products and is not comparable to the micro-controller which were specifically designed for seismic data acquisition purposes. The key point to note is the feasibility of the proposed idea and the features presented can be extended to the present state of the art technology